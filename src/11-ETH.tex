%!TEX root = ../out/12-ETH-SL.tex

\begin{document}

\mytitle{Exponential Time Hypothesis}{Exponential Time Hypothesis}

\section{SAT and k-SAT}

\begin{frame}
	\frametitle{SAT}
	
  \pbDef{\SAT}{A propositional formula $F$ in conjunctive normal form (CNF)}{$n=|\var(F)|$, the number of variables in $F$}{Is there an assignment to $\var(F)$ satisfying all clauses of $F$?}
  \pbDef{$k$-\SAT}{A CNF formula $F$ where each clause has length at most $k$}{$n=|\var(F)|$, the number of variables in $F$}{Is there an assignment to $\var(F)$ satisfying all clauses of $F$?}
  
  \noindent
  \textbf{Example}:
  \begin{align*}
  	(\textcolor{Green}{x_1} \vee \textcolor{red}{x_2}) \wedge (\neg \textcolor{red}{x_2} \vee \textcolor{Green}{x_3} \vee \neg \textcolor{red}{x_4}) \wedge
  	(\textcolor{Green}{x_1} \vee \textcolor{red}{x_4}) \wedge (\neg \textcolor{Green}{x_1} \vee \neg \textcolor{Green}{x_3} \vee \neg \textcolor{red}{x_4})
  \end{align*}
\end{frame}

\begin{frame}
	\frametitle{Algorithms for \SAT}
	
	\begin{itemize}
		\item Brute-force: $O^*(2^n)$
		\pause
		\item ... after $>50$ years of SAT solving\\
		      (SAT association, SAT conference, JSAT journal, annual SAT competitions, ...)
		\pause
		\item fastest known algorithm for SAT: $O^*(2^{n\cdot(1-1/O(\log m/n))})$, where $m$ is the number of clauses \cite{CalabroIP06} \cite{DantsinH09}
		\item However: no $O^*(1.9999^n)$ time algorithm is known
		\item \lecturenotes{fastest }known algorithms for 3-SAT: $O^*(1.3280^n)$ deterministic \cite{Liu18} and $O^*(1.3070^n)$ randomized \cite{HansenKZZ19}
		\pause
		\medskip
		\item Could it be that 3-SAT cannot be solved in $2^{o(n)}$ time?
		\item Could it be that SAT cannot be solved in $O^*((2-\epsilon)^n)$ time for any $\epsilon>0$?
	\end{itemize}
\end{frame}

\section{Subexponential time algorithms}

\begin{frame}
	\frametitle{NP-hard problems in subexponential time?}
	
	\begin{itemize}
		\item Are there any NP-hard problems that can be solved in $2^{o(n)}$ time?
		\pause
		\item \textbf{Yes}. For example, \textsc{Independent Set} is \NP-complete even when the input graph is planar (can be drawn in the plane without edge crossings). Planar graphs have treewidth $O(\sqrt{n})$ and tree decompositions of that width can be found in polynomial time (``Planar separator theorem'' \mycite{Lipton, Tarjan, 1979}). Using a tree decomposition based algorithm, \textsc{Independent Set} can be solved in $2^{O(\sqrt{n})}$ time on planar graphs.
	\end{itemize}
\end{frame}

\section{ETH and SETH}

\begin{frame}
 \slides{\frametitle{ETH and SETH}}
 \begin{definition}
  For each $k\ge 3$, define \alert{$\delta_k$} to be the infinimum\footnote[frame]{The infinimum of a set of numbers is the largest number that is smaller or equal to each number in the set. E.g., the infinimum of $\{\varepsilon \in \mathbb{R} : \varepsilon>0\}$ is $0$.} of the set of constants $c$ such that $k$-SAT can be solved in $O^*(2^{c\cdot n})$ time.
 \end{definition}
 
 \begin{conjecture}[Exponential Time Hyphothesis (ETH)]
  $\delta_3>0$.
 \end{conjecture}

 \begin{conjecture}[Strong Exponential Time Hyphothesis (SETH)]
  $\lim_{k\rightarrow \infty}\delta_k=1$.
 \end{conjecture}
 
 \noindent
 \textbf{Notes}: (1) ETH $\Rightarrow$ 3-SAT cannot be solved in $2^{o(n)}$ time.\\
 SETH $\Rightarrow$ SAT cannot be solved in $O^*((2-\epsilon)^n)$ time for any $\epsilon>0$.

\end{frame}

\section{Algorithmic lower bounds based on ETH}

\begin{frame}
 \slides{\frametitle{Algorithmic lower bounds based on ETH}}
 
 \begin{itemize}
  \item Suppose ETH is true
  \item Can we infer lower bounds on the running time needed to solve other problems?
 \pause
  \item Suppose there is a polynomial-time reduction from 3-SAT to a graph problem $\Pi$, which constructs an equivalent instance where the number of vertices of the output graph equals the number of variables of the input formula, $|V|=|\var(F)|$.
  \item Using the reduction, we can conclude that, if $\Pi$ has an $O^*(2^{o(|V|)})$ time algorithm, then 3-SAT has an $O^*(2^{o(|\var(F)|)})$ time algorithm, contradicting ETH.\\
  \item Therefore, we conclude that $\Pi$ has no $O^*(2^{o(|V|)})$ time algorithm unless ETH fails.
 \end{itemize}
\end{frame}

\begin{frame}
 \frametitle{Sparsification Lemma}
 
 \noindent
 \textbf{Issue}: Many reductions from 3-SAT create a number of vertices / variables / elements that are related to the number of \alert{clauses} of the 3-SAT instance.
 
 \pause
 \begin{theorem}[Sparsification Lemma, \cite{ImpagliazzoPZ01}]
  For each $\varepsilon > 0$ and positive integer $k$, there is a $O^*(2^{\varepsilon \cdot n})$ time algorithm that takes as input a $k$-CNF formula $F$ with $n$ variables and outputs an equivalent formula $F'=\bigvee_{i=1}^t F_i$ that is a disjunction of $t\le 2^{\varepsilon n}$ formulas $F_i$ with $\var(F_i)=\var(F)$ and $|\cla(F_i)| = O(n)$.
 \end{theorem}

\end{frame}

\begin{frame}
 \frametitle{3-SAT with a linear number of clauses}
 
 \begin{corollary}
  ETH $\Rightarrow$ 3-SAT cannot be solved in $O^*(2^{o(n+m)})$ time where $m$ denotes the number of clauses of $F$.
 \end{corollary}
 
 \noindent
 \textbf{Observation}: Let $A$, $B$ be parameterized problems and $f$, $g$ be non-decreasing functions.\\
 Suppose there is a polynomial-time reduction from $A$ to $B$ such that if the parameter of an instance of $A$ is $k$, then the parameter of the constructed instance of $B$ is at most $g(k)$.\\
 Then an $O^*(2^{o(f(k))})$ time algorithm for $B$ implies an $O^*(2^{o(f(g(k)))})$ time algorithm for $A$.

\end{frame}

\begin{frame}
 \frametitle{More general reductions are possible}
 
 \begin{definition}[SERF-reduction]
  A \alert{SubExponential Reduction Family} from a parameterized problem $A$ to a parameterized problem $B$ is a family of \alert{Turing reductions} from $A$ to $B$ (i.e., an algorithm for $A$, making queries to an \alert{oracle} for $B$ that solves any instance for $B$ in constant time) for each $\varepsilon >0$ such that
  \begin{itemize}
   \item for every instance $I$ for $A$ with parameter $k$, the running time is $O^*(2^{\varepsilon k})$, and
   \item for every query $I'$ to $B$ with parameter $k'$, we have that $k' \in O(k)$ and $|I'|=|I|^{O(1)}$.
  \end{itemize}
 \end{definition}

 \noindent
 \textbf{Note}: If $A$ is SERF-reducible to $B$ and $A$ has no $2^{o(k)}$ time algorithm, then $B$ has no $2^{o(k')}$ time algorithm.
\end{frame}


\begin{frame}
 \frametitle{Vertex Cover has no subexponential algorithm}

 \pause
 \noindent
 Polynomial-time reduction from 3-SAT.\newline
 For simplicity, assume all clauses have length $3$.\newline
 3-CNF Formula $F=(\textcolor{Green}{u} \vee \textcolor{red}{v} \vee \neg \textcolor{Green}{y}) \wedge 
	(\neg \textcolor{Green}{u}\vee \textcolor{Green}{y}\vee \textcolor{Green}{z}) \wedge
	(\neg \textcolor{red}{v}\vee \textcolor{Green}{w}\vee \textcolor{red}{x}) \wedge
	(\textcolor{red}{x}\vee \textcolor{Green}{y}\vee \neg \textcolor{Green}{z})$ %$F=\{C,D,E,F,G\}$ where  
%  $C= \{u, v, \neg y\}$, 
%  $D= \{\neg u, z\}$, 
%  $E= \{\neg v, w\}$, 
%  $F= \{\neg w, x\}$, 
%  $G= \{x, y, \neg z\}$.

\pause
\begin{center}
\begin{tikzpicture}[scale=.8]
 %\tikzstyle{every node}=[circle, inner sep=0pt]
 \pgfsetbaseline{-1.5cm}

\draw (0,0) node[vertex,label=above:$\neg u$] (nu) {};
\draw (1,0) node[selected,label=above:$u$] (u) {};
\draw (2,0) node[selected,label=above:$\neg v$] (nv) {};
\draw (3,0) node[vertex,label=above:$v$] (v) {};
\draw (4,0) node[vertex,label=above:$\neg w$] (nw) {};
\draw (5,0) node[selected,label=above:$w$] (w) {};
\draw (6,0) node[selected,label=above:$\neg x$] (nx) {};
\draw (7,0) node[vertex,label=above:$x$] (x) {};
\draw (8,0) node[vertex,label=above:$\neg y$] (ny) {};
\draw (9,0) node[selected,label=above:$y$] (y) {};
\draw (10,0) node[vertex,label=above:$\neg z$] (nz) {};
\draw (11,0) node[selected,label=above:$z$] (z) {};

\draw (0.75,-4) node[vertex] (Cu) {};
\draw (1.25,-3) node[selected] (Cv) {};
\draw (1.75,-4) node[selected] (Cy) {};
\draw (3.25,-4) node[selected] (Du) {};
\draw (3.75,-3) node[selected] (Dy) {};
\draw (4.25,-4) node[vertex] (Dz) {};
\draw (5.75,-4) node[vertex] (Ev) {};
\draw (6.25,-3) node[selected] (Ew) {};
\draw (6.75,-4) node[selected] (Ex) {};
\draw (8.25,-4) node[selected] (Fx) {};
\draw (8.75,-3) node[vertex] (Fy) {};
\draw (9.25,-4) node[selected] (Fz) {};

\draw[thick] 
(nu)--(u) 
(nv)--(v) 
(nw)--(w) 
(nx)--(x) 
(ny)--(y) 
(nz)--(z);

\draw[thick] 
(Cu)--(Cv)--(Cy)--(Cu) 
(Du)--(Dy)--(Dz)--(Du) 
(Ev)--(Ew)--(Ex)--(Ev)
(Fx)--(Fy)--(Fz)--(Fx);

\draw[thick]
(u)--(Cu) (Cv)--(v) (ny)--(Cy)
(nu)--(Du) (Dz)--(z) (y)--(Dy)
(nv)--(Ev) (Ew)--(w) (Ex)--(x)
(x)--(Fx) (Fy)--(y) (nz)--(Fz);

  \end{tikzpicture}
\end{center}

\noindent
For a 3-CNF formula with $n$ variables and $m$ clauses, we create a \textsc{Vertex Cover} instance with
$|V| = 2n+3m$, $|E| = n+6m$, and $k=n+2m$.

\end{frame}

\begin{frame}
 \slides{\frametitle{Vertex Cover has no subexponential algorithm II}}

% \begin{theorem}
%  ETH $\Rightarrow$ \textsc{Vertex Cover} has no $2^{o(|V|)}$ time algorithm.
% \end{theorem}
% \begin{theorem}
%  ETH $\Rightarrow$ \textsc{Vertex Cover} has no $2^{o(|E|)}$ time algorithm.
% \end{theorem}
% \begin{theorem}
%  ETH $\Rightarrow$ \textsc{Vertex Cover} has no $2^{o(k)}$ time algorithm.
% \end{theorem}
 \begin{theorem}
	ETH $\Rightarrow$ \textsc{Vertex Cover} has no $2^{o(|V|+|E|+k)}$ time algorithm.
 \end{theorem}

\end{frame}

\section{Algorithmic lower bounds based on SETH}

\begin{frame}{Hitting Set}
	
	\noindent
	\textbf{Recall}: A \alert{hitting set} of a set system $\mathcal{S}=(V,H)$ is a subset $X$ of $V$ such that $X$ contains at least one element of each set in $H$, i.e., $X\cap Y \ne \emptyset$ for each $Y\in H$. 
	
	\pbDef{\textsf{elts}-\textsc{Hitting Set}}{A set system $\mathcal{S}=(V,H)$ and an integer $k$}{$n=|V|$}{Does $\mathcal{S}$ have a hitting set of size at most $k$?}
	
	\begin{center}
		\begin{tikzpicture} %[scale=1.2]
		%\begin{scope}[shape=circle,minimum size=0.2cm]
		%\tikzstyle{every node}=[fill=black]
		\node[vertex]       (a) at (0,1) {};
		\node[vertex]                 (b) at (1,1) {};
		\node[vertex]                 (c) at (2,1) {};
		\node[selected]                 (d) at (3,1) {};
		\node[selected]                 (e) at (0,0) {};
		\node[vertex]       (f) at (1,0) {};
		\node[selected]                 (g) at (2,0) {};
		\node[vertex]       (h) at (3,0) {};
		%\end{scope}
		%\draw[line width=2pt]     (a) -- (b) -- (e) -- (h) -- (g) -- (f) -- (d) -- (g) -- (e) -- (a) -- (c) -- (d) -- (a);
		\draw plot[smooth cycle] coordinates{(0.7,1.3) (2.3,1.3) (2.3,-0.3) (0.7,-0.3)};
		\draw plot[smooth cycle] coordinates{(-0.3,1.3) (0.3,1.3) (0.3,-0.3) (-0.3,-0.3)};
		\draw plot[smooth cycle] coordinates{(1.7,0.3) (3.3,0.3) (3.3,-0.3) (1.7,-0.3)};
		\draw plot[smooth cycle] coordinates{(1.7,1.3) (3.3,1.3) (3.3,-0.3) (2.7,-0.3) (1.7,0.7)};
		\draw plot[smooth cycle] coordinates{(1.3,1.3) (1.3,-0.3) (-0.3,-0.3) (-0.3,0.3) (0.7,1.3)};
		
		\end{tikzpicture}
		
	\end{center}
\end{frame}

\begin{frame}
 \frametitle{SETH-lower bound for Hitting Set}
 
 CNF Formula $F=(\textcolor{Green}{u} \vee \textcolor{red}{v} \vee \neg \textcolor{Green}{y}) \wedge 
	(\neg \textcolor{Green}{u}\vee \textcolor{Green}{y}\vee \textcolor{Green}{z}) \wedge
	(\neg \textcolor{red}{v}\vee \textcolor{Green}{w}\vee \textcolor{red}{x}) \wedge
	(\textcolor{red}{x}\vee \textcolor{Green}{y}\vee \neg \textcolor{Green}{z})$ %$F=\{C,D,E,F,G\}$ where  
%  $C= \{u, v, \neg y\}$, 
%  $D= \{\neg u, z\}$, 
%  $E= \{\neg v, w\}$, 
%  $F= \{\neg w, x\}$, 
%  $G= \{x, y, \neg z\}$.

Inidence graph of equivalent Hitting Set instance:

\begin{center}
\begin{tikzpicture}[scale=.85]
 \tikzstyle{every node}=[inner sep=1pt]
 \pgfsetbaseline{-1.5cm}

\draw (0,0) node[vertex,label=above left:$\neg u$] (nu) {};
\draw (1,0) node[selected,label=above right:$u$] (u) {};
\draw (2,0) node[selected,label=below left:$\neg v$] (nv) {};
\draw (3,0) node[vertex,label=below right:$v$] (v) {};
\draw (4,0) node[vertex,label=above left:$\neg w$] (nw) {};
\draw (5,0) node[selected,label=above right:$w$] (w) {};
\draw (6,0) node[selected,label=below left:$\neg x$] (nx) {};
\draw (7,0) node[vertex,label=below right:$x$] (x) {};
\draw (8,0) node[vertex,label=above left:$\neg y$] (ny) {};
\draw (9,0) node[selected,label=above right:$y$] (y) {};
\draw (10,0) node[vertex,label=below left:$\neg z$] (nz) {};
\draw (11,0) node[selected,label=below right:$z$] (z) {};

\draw (0.5,2) node[vertex] (cu) {};
\draw (2.5,2) node[vertex] (cv) {};
\draw (4.5,2) node[vertex] (cw) {};
\draw (6.5,2) node[vertex] (cx) {};
\draw (8.5,2) node[vertex] (cy) {};
\draw (10.5,2) node[vertex] (cz) {};

\draw (1.25,-3) node[vertex] (C) {};
\draw (3.75,-3) node[vertex] (D) {};
\draw (6.25,-3) node[vertex] (E) {};
\draw (8.75,-3) node[vertex] (F) {};

\node at (-2,2) {sets};
\node at (-2,0) {elts};
\node at (-2,-3) {sets};


\draw[thick] 
(nu)--(cu)--(u) 
(nv)--(cv)--(v) 
(nw)--(cw)--(w) 
(nx)--(cx)--(x) 
(ny)--(cy)--(y) 
(nz)--(cz)--(z);

\draw[thick]
(u)--(C) (C)--(v) (ny)--(C)
(nu)--(D) (D)--(z) (y)--(D)
(nv)--(E) (E)--(w) (E)--(x)
(x)--(F) (F)--(y) (nz)--(F);

  \end{tikzpicture}
\end{center}

\noindent
For a CNF formula with $n$ variables and $m$ clauses, we create a \textsc{Hitting Set} instance with
$|V| = 2n$ and $k=n$.
\end{frame}


\begin{frame}
 \slides{\frametitle{SETH-lower bound for Hitting Set}}

 \begin{theorem}
  SETH $\Rightarrow$ \textsc{Hitting Set} has no $O^*((2-\varepsilon)^{|V|/2})$ time algorithm for any $\varepsilon>0$.
 \end{theorem}
 \noindent
 \textbf{Note}: With a more ingenious reduction, one can show that \textsc{Hitting Set} has no $O^*((2-\varepsilon)^{|V|})$ time algorithm for any $\varepsilon>0$ under SETH \cite{CyganDLMNOPSW16}.
\end{frame}

\section{Further Reading}

\begin{frame}
 \slides{\frametitle{Further Reading}}

 \begin{itemize}
   \item Chapter 14, \emph{Lower bounds based on the Exponential-Time Hypothesis} in \cite{CyganFKL+15}
%   Marek Cygan, Fedor V.\ Fomin, \L ukasz Kowalik, Daniel Lokshtanov, D{\'a}niel Marx, Marcin Pilipczuk, Micha\l Pilipczuk, and Saket Saurabh. Parameterized Algorithms. Springer, 2015.
 	\item Section 11.3, \emph{Subexponential Algorithms and ETH} in \cite{FominK10}
% 	Fedor V. Fomin and Dieter Kratsch. Exact Exponential Algorithms. Springer, 2010.
 	\item Section 29.5, \emph{The Sparsification Lemma} in \cite{DowneyF13}
% 	Rodney G. Downey and Michael R. Fellows. Fundamentals of Parameterized Complexity. Springer, 2013.
 \end{itemize}

\end{frame}

\begin{frame}[t, allowframebreaks]
	\slides{\frametitle{References}}
	\printbibliography
\end{frame}


\end{document}
