%!TEX root = ../out/06-W-SL.tex

\begin{document}

\title[Parameterized intractability]
{Parameterized intractability: the W-hierarchy}

\begin{frame}
  \titlepage
\end{frame}

\lecturenotes{\maketitle}

\begin{frame}
 \slides{\frametitle{Outline}}
 \tableofcontents
\end{frame}

\section{Parameterized Complexity Theory}

\begin{frame}
	\frametitle{Main Parameterized Complexity Classes}
	
	\noindent
	$n$: instance size\newline
	$k$: parameter
	
	\medskip
	\noindent
	$\P$: class of problems that can be solved in $n^{O(1)}$ time\newline
	$\FPT$: class of parameterized problems that can be solved in $f(k) \cdot n^{O(1)}$ time\newline
	$\W[\cdot]$: parameterized intractability classes\newline
	$\XP$: class of parameterized problems that can be solved in $f(k) \cdot n^{g(k)}$ time\\ $\qquad$ (``polynomial when $k$ is a constant'')
	%
	\slides{\medskip}
	\begin{align*}
	\P \subseteq \FPT \subseteq \W[1] \subseteq \W[2] \dots \subseteq \W[P]\subseteq \XP
	\end{align*}
	%\slides{\medskip}
	
	%\noindent
	%\textbf{Known}: If $\FPT = \W[1]$, then the Exponential Time Hypothesis fails,
	%i.e. 3-\textsc{Sat} can be solved in $2^{o(n)}$ time, where $n$ is the number of variables.
	
	\smallskip
	\noindent
	\textbf{Note}: We assume that $f$ is \alert{computable} and \alert{non-decreasing}.
	
\end{frame}


\begin{frame}{Polynomial-time reductions for parameterized problems?}

 \noindent
 A \alert{vertex cover} in a graph $G=(V,E)$ is a subset of vertices $S\subseteq V$ such that every edge of $G$ has an endpoint in $S$.
 \pbDef{\textsc{Vertex Cover}}{Graph $G$, integer $k$}{$k$}{Does $G$ have a vertex cover of size $k$?}
%
 \noindent
 An \alert{independent set} in a graph $G=(V,E)$ is a subset of vertices $S\subseteq V$ such that there is no edge $uv\in E$ with $u,v\in S$.
 \pbDef{\textsc{Independent Set}}{Graph $G$, integer $k$}{$k$}{Does $G$ have an independent set of size $k$?}
 
 \pause
 \begin{itemize}
  \item We know: \textsc{Independent Set} $\le_\P$ \textsc{Vertex Cover}
  \item However: \textsc{Vertex Cover} $\in \FPT$ but \textsc{Independent Set} is not known to be in $\FPT$
 \end{itemize}

\end{frame}

\begin{frame}
 \frametitle{We will need another type of reductions}
 
 \begin{itemize}
  \item Issue with polynomial-time reductions: parameter can change arbitrarily.
  \pause
  \item We will want the reduction to produce an instance where the parameter is bounded by a function of the parameter of the original instance.
  \smallskip
  \pause
  \item Also: we can allow the reduction to take \FPT\ time instead of only polynomial time.
 \end{itemize}
 
\end{frame}


\subsection{Parameterized reductions}


\begin{frame}{Parameterized reduction}
 \begin{definition}
 	A \alert{parameterized reduction} from a parameterized decision problem $\Pi_1$ to a parameterized decision problem $\Pi_2$ is an algorithm, which, for any instance $I$ of $\Pi_1$ with parameter $k$ produces an instance $I'$ of $\Pi_2$ with parameter $k'$ such that
 	\begin{itemize}
 		\item $I$ is a \Yes-instance for $\Pi_1$ $\Leftrightarrow$ $I'$ is a \Yes-instance for $\Pi_2$,
 		\item there exists a computable function $g$ such that $k'\le g(k)$, and
 		\item there exists a computable function $f$ such that the running time of the algorithm is $f(k) \cdot |I|^{O(1)}$.
  	\end{itemize}
 	If there exists a parameterized reduction from $\Pi_1$ to $\Pi_2$, we write $\Pi_1 \le_{\FPT} \Pi_2$.
 \end{definition}
 
 \noindent
 \textbf{Note}: We can assume that $f$ and $g$ are non-decreasing.
\end{frame}


\begin{frame}{New FPT algorithms via reductions}
	
	\begin{lemma}
		If $\Pi_1, \Pi_2$ are parameterized decision problems such that $\Pi_1 \le_{\FPT} \Pi_2$, then\\ $\Pi_2 \in \FPT$ implies $\Pi_1\in \FPT$.
	\end{lemma}
	\begin{proof}[Proof sketch]
		To obtain an \FPT\ algorithm for $\Pi_1$, perform the reduction and then use an \FPT\ algorithm for $\Pi_2$ on the resulting instance.
	\end{proof}
	
\end{frame}


\subsection{Parameterized complexity classes}

\begin{frame}{Boolean Circuits}
	\begin{definition}
		A \alert{Boolean circuit} is a directed acyclic graph with the nodes labeled as follows:
		\begin{itemize}
			 \item every node of in-degree $0$ is an \alert{input node},
			 \item every node with in-degree $1$ is a \alert{negation node} ($\neg$), and
			 \item every node with in-degree $\ge 2$ is either an \alert{AND-node} ($\wedge$) or an \alert{OR-node} ($\vee$).
		\end{itemize}
		Moreover, exactly one node with out-degree 0 is also labeled the \alert{output node}.\newline
		The \alert{depth} of the circuit is the maximum length of a directed path from an input node to the output node.\newline
		The \alert{weft} of the circuit is the maximum number of nodes with in-degree $\ge 3$ on a directed path from an input node to the output node. 
	\end{definition}
\end{frame}

\begin{frame}{Example}
	\begin{center}
		\begin{tikzpicture}
			\node[vertex,label=below:$a$] (a) at (0,0) {};
			\node[vertex,label=below:$b$] (b) at (1,0) {};
			\node[vertex,label=below:$c$] (c) at (2,0) {};
			\node[vertex,label=below:$d$] (d) at (3,0) {};
			\node[vertex,label=below:$e$] (e) at (4,0) {};
			
			\node[gate] (an) at (0,1) {$\neg$};
			\node[gate] (bn) at (1,1) {$\neg$};
			\node[gate] (cn) at (2,1) {$\neg$};
			\node[gate] (dn) at (3,1) {$\neg$};
			\node[gate] (en) at (4,1) {$\neg$};
			
			\node[gate] (ab) at (-1,2) {$\vee$};
			\node[gate] (bc) at (0,2) {$\vee$};
			\node[gate] (ac) at (1,2) {$\vee$};
			\node[gate] (bd) at (2,2) {$\vee$};
			\node[gate] (cd) at (3,2) {$\vee$};
			\node[gate] (ce) at (4,2) {$\vee$};
			\node[gate] (de) at (5,2) {$\vee$};
			
			\node[gate] (o) at (2,3) {$\wedge$};
			\node (out) at (2,3.5) {out};
			
			\draw[->] (a)--(an);
			\draw[->] (b)--(bn);
			\draw[->] (c)--(cn);
			\draw[->] (d)--(dn);
			\draw[->] (e)--(en);
			\draw[->] (an)--(ab);
			\draw[->] (an)--(ac);
			\draw[->] (bn)--(ab);
			\draw[->] (bn)--(bc);
			\draw[->] (bn)--(bd);
			\draw[->] (cn)--(ac);
			\draw[->] (cn)--(bc);
			\draw[->] (cn)--(cd);
			\draw[->] (cn)--(ce);
			\draw[->] (dn)--(bd);
			\draw[->] (dn)--(cd);
			\draw[->] (dn)--(de);
			\draw[->] (en)--(ce);
			\draw[->] (en)--(de);
			\draw[->] (ab)--(o);
			\draw[->] (bc)--(o);
			\draw[->] (ac)--(o);
			\draw[->] (bd)--(o);
			\draw[->] (cd)--(o);
			\draw[->] (ce)--(o);
			\draw[->] (de)--(o); 
		\end{tikzpicture}
	\end{center}
	A depth-3, weft-1 Boolean circuit with inputs $a,b,c,d,e$.
\end{frame}

\begin{frame}{Weighted Circuit Satisfiability}
	
	\noindent
	Given an assignment of Boolean values to the input gates, the circuit determines Boolean values at each node in the obvious way.
	
	\noindent
	If the value of the output node is $1$ for an input assignment, we say that this assignment \alert{satisfies} the circuit.
	
	\noindent
	The \alert{weight} of an assignment is its number of $1$s.
	
	\pbDef{\textsc{Weighted Circuit Satisfiability} (WCS)}{A Boolean circuit $C$, an integer $k$}{$k$}{Is there an assignment with weight $k$ that satisfies $C$?}
	
	\noindent
	\textbf{Exercise}: Show that \textsc{Weighted Circuit Satisfiability} $\in \XP$.
	 
\end{frame}

\begin{frame}{WCS for special circuits}
	\begin{definition}
	The class of circuits $\mathcal{C}_{t,d}$ contains the circuits with weft $\le t$ and depth $\le d$.
	\end{definition}
	
	\noindent
	For any class of circuits $\mathcal{C}$, we can define the following problem.
	\pbDef{WCS[$\mathcal{C}$]}{A Boolean circuit $C\in \mathcal{C}$, an integer $k$}{$k$}{Is there an assignment with weight $k$ that satisfies $C$?}
\end{frame}

\begin{frame}{W classes}
	
	\begin{definition}[W-hierarchy]
		Let $t\in \{1,2,\dots \}$.
		A parameterized problem $\Pi$ is in the parameterized complexity class $\W[t]$ if there exists a parameterized reduction from $\Pi$ to WCS[$\mathcal{C}_{t,d}$] for some constant $d\ge 1$.
	\end{definition}
	
\end{frame}

\begin{frame}{Independent Set and Dominating Set}
	
	\begin{theorem}
		\textsc{Independent Set} $\in \W[1]$.
	\end{theorem}
	\begin{theorem}
		\textsc{Dominating Set} $\in \W[2]$.
	\end{theorem}
	
	\noindent
	\textbf{Recall}: A \alert{dominating set} of a graph $G=(V,E)$ is a set of vertices $S\subseteq V$ such that $N_G[S]=V$.
        \pbDef{\textsc{Dominating Set}}{A graph $G=(V,E)$ and an integer $k$}{$k$}{Does $G$ have a dominating set of size at most $k$?}
\end{frame}

\begin{frame}{``Proof'' by picture}

	\noindent
	Parameterized reductions from \textsc{Independent Set} to WCS[$\mathcal{C}_{1,3}$] and from \textsc{Dominating Set} to WCS[$\mathcal{C}_{2,2}$].
	\begin{center}
		\begin{tikzpicture}[scale=0.8]
		 \node[vertex,label=above:$a$] (a) at (1,2) {};
		 \node[vertex,label=left:$b$] (b) at (0,1) {};
		 \node[vertex,label=right:$c$] (c) at (2,1) {};
		 \node[vertex,label=below:$d$] (d) at (0,-0.5) {};
		 \node[vertex,label=below:$e$] (e) at (2,-0.5) {};
		 
		 \draw (b)--(a)--(c)--(e)--(d)--(c)--(b)--(d);
		\end{tikzpicture}
		\begin{tikzpicture}[scale=0.8]
		\node[vertex,label=below:$a$] (a) at (0,-0.5) {};
		\node[vertex,label=below:$b$] (b) at (1,-0.5) {};
		\node[vertex,label=below:$c$] (c) at (2,-0.5) {};
		\node[vertex,label=below:$d$] (d) at (3,-0.5) {};
		\node[vertex,label=below:$e$] (e) at (4,-0.5) {};
		
		\node[gate] (an) at (0,0.5) {$\neg$};
		\node[gate] (bn) at (1,0.5) {$\neg$};
		\node[gate] (cn) at (2,0.5) {$\neg$};
		\node[gate] (dn) at (3,0.5) {$\neg$};
		\node[gate] (en) at (4,0.5) {$\neg$};
		
		\node[gate] (ab) at (-1,2) {$\vee$};
		\node[gate] (bc) at (0,2) {$\vee$};
		\node[gate] (ac) at (1,2) {$\vee$};
		\node[gate] (bd) at (2,2) {$\vee$};
		\node[gate] (cd) at (3,2) {$\vee$};
		\node[gate] (ce) at (4,2) {$\vee$};
		\node[gate] (de) at (5,2) {$\vee$};
		
		\node[gate] (o) at (2,3) {$\wedge$};
		\node (out) at (2,3.5) {out};
		
		\draw[->] (a)--(an);
		\draw[->] (b)--(bn);
		\draw[->] (c)--(cn);
		\draw[->] (d)--(dn);
		\draw[->] (e)--(en);
		\draw[->] (an)--(ab);
		\draw[->] (an)--(ac);
		\draw[->] (bn)--(ab);
		\draw[->] (bn)--(bc);
		\draw[->] (bn)--(bd);
		\draw[->] (cn)--(ac);
		\draw[->] (cn)--(bc);
		\draw[->] (cn)--(cd);
		\draw[->] (cn)--(ce);
		\draw[->] (dn)--(bd);
		\draw[->] (dn)--(cd);
		\draw[->] (dn)--(de);
		\draw[->] (en)--(ce);
		\draw[->] (en)--(de);
		\draw[->] (ab)--(o);
		\draw[->] (bc)--(o);
		\draw[->] (ac)--(o);
		\draw[->] (bd)--(o);
		\draw[->] (cd)--(o);
		\draw[->] (ce)--(o);
		\draw[->] (de)--(o); 
		\end{tikzpicture}
		\begin{tikzpicture}[scale=0.8]
		\node[vertex,label=below:$a$] (a) at (0,-1) {};
		\node[vertex,label=below:$b$] (b) at (1,-1) {};
		\node[vertex,label=below:$c$] (c) at (2,-1) {};
		\node[vertex,label=below:$d$] (d) at (3,-1) {};
		\node[vertex,label=below:$e$] (e) at (4,-1) {};
		
		\node[gate] (an) at (0,1) {$\vee$};
		\node[gate] (bn) at (1,1) {$\vee$};
		\node[gate] (cn) at (2,1) {$\vee$};
		\node[gate] (dn) at (3,1) {$\vee$};
		\node[gate] (en) at (4,1) {$\vee$};
		
		\node[gate] (o) at (2,2) {$\wedge$};
		\node (out) at (2,2.5) {out};
		
		\draw[->] (a)--(an);
		\draw[->] (b)--(bn);
		\draw[->] (c)--(cn);
		\draw[->] (d)--(dn);
		\draw[->] (e)--(en);
		\draw[->] (a)--(bn);
		\draw[->] (b)--(cn);
		\draw[->] (c)--(dn);
		\draw[->] (d)--(en);
		\draw[->] (a)--(cn);
		\draw[->] (b)--(dn);
		\draw[->] (c)--(en);
		\draw[->] (c)--(an);
		\draw[->] (d)--(bn);
		\draw[->] (e)--(cn);
		\draw[->] (b)--(an);
		\draw[->] (c)--(bn);
		\draw[->] (d)--(cn);
		\draw[->] (e)--(dn);
		
		\draw[->] (an)--(o);
		\draw[->] (bn)--(o);
		\draw[->] (cn)--(o);
		\draw[->] (dn)--(o);
		\draw[->] (en)--(o); 
		\end{tikzpicture}	
	\end{center}
	Setting an input node to $1$ corresponds to adding the corresponding vertex to the independent set / dominating set.

\end{frame}

\begin{frame}{W-hardness}
	
	\begin{definition}
		Let $t\in \{1,2,\dots\}$.\\
		A parameterized decision problem $\Pi$ is \alert{$\W[t]$-hard} if for every parameterized decision problem $\Pi'$ in $\W[t]$, there is a parameterized reduction from $\Pi'$ to $\Pi$.\\
		$\Pi$ is \alert{$\W[t]$-complete} if $\Pi \in \W[t]$ and $\Pi$ is $\W[t]$-hard.
	\end{definition}
	
	\pause
	\begin{theorem}[\cite{DowneyF95II}]
		\textsc{Independent Set} is $\W[1]$-complete.
	\end{theorem}
	
	\begin{theorem}[\cite{DowneyF95I}]
		\textsc{Dominating Set} is $\W[2]$-complete.
	\end{theorem}
\end{frame}

\begin{frame}{Proving W-hardness}
	
	To show that a parameterized decision problem $\Pi$ is $\W[t]$-hard:
	\begin{itemize}
		\item Select a $\W[t]$-hard problem $\Pi'$
		\item Show that $\Pi' \le_{\FPT} \Pi$ by designing a parameterized reduction from $\Pi'$ to $\Pi$
		\begin{itemize}
			\item Design an algorithm, that, for any instance $I'$ of $\Pi'$ with parameter $k'$, produces an equivalent instance $I$ of $\Pi$ with parameter $k$
			\item Show that $k$ is upper bounded by a function of $k'$
			\item Show that there exists a function $f$ such that the running time of the algorithm is $f(k') \cdot |I'|^{O(1)}$
		\end{itemize}
	\end{itemize}
\end{frame}


\section{Case study}

\begin{frame}{Clique}
	
 \noindent
 A \alert{clique} in a graph $G=(V,E)$ is a subset of its vertices $S\subseteq V$ such that every two vertices from $S$ are adjacent in $G$.
 
 \pbDef{\textsc{Clique}}
 {Graph $G=(V,E)$, integer $k$}
 {$k$}
 {Does $G$ have a clique of size $k$?}
 
 \begin{center}
 	\begin{tikzpicture}[scale=1]
 	
 	\draw (0,0) node[vertex] (v1) {};
 	\draw (1,0) node[selected] (v2) {};
 	\draw (2,0) node[selected] (v3) {};
 	\draw (3,0) node[vertex] (v4) {};
 	\draw (4,0) node[vertex] (v5) {};
 	\draw (0,1) node[vertex] (v6) {};
 	\draw (1,1) node[selected] (v7) {};
 	\draw (3,1) node[vertex] (v8) {};
 	\draw (4,1) node[vertex] (v9) {};
 	
 	\draw[very thick] (v6)--(v1)--(v7)--(v8)--(v4)--(v5)--(v9) (v7)--(v2)--(v3);
 	\draw[very thick] (v6)--(v7)--(v3) (v8)--(v9)--(v4);
 	\end{tikzpicture}
 \end{center}
 
 \begin{itemize}
 	\item We will show that \textsc{Clique} is $\W[1]$-hard by a parameterized reduction from \textsc{Independent Set}.
 \end{itemize}
\end{frame}

\begin{frame}
	\slides{\frametitle{Clique is W[1]-hard}}
	
	\begin{lemma}
		\textsc{Independent Set} $\le_{\FPT}$ \textsc{Clique}.
	\end{lemma}
	\begin{proof}
		Given any instance $(G=(V,E),k)$ for \textsc{Independent Set}, we need to describe an \FPT\ algorithm that constructs an equivalent instance $(G',k')$ for \textsc{Clique} such that $k'\le g(k)$ for some computable function $g$.
		
		\pause
		\noindent
		\textbf{Construction.}
		Set $k' \leftarrow k$ and $G' \leftarrow %\overline{G}$, where
		\overline{G} = (V,\{uv : u,v \in V, u\ne v, uv\notin E\})$.
		
		\pause
		\noindent
		\textbf{Equivalence.}
		We need to show that $(G,k)$ is a \Yes-instance for \textsc{Independent Set} if and only if $(G',k')$ is a \Yes-instance for \textsc{Clique}.\newline
		\pause
		$(\Rightarrow)$: Let $S$ be an independent set of size $k$ in $G$. For every two vertices $u,v\in S$, we have that $uv \notin E$. Therefore, $uv\in E(\overline{G})$ for every two vertices in $S$. We conclude that $S$ is a clique of size $k$ in $\overline{G}$.\newline
		\pause
		$(\Leftarrow)$: Let $S$ be a clique of size $k$ in $\overline{G}$. By a similar argument, $S$ is an independent set of size $k$ in $G$.
		
		\pause
		\noindent
		\textbf{Parameter.}
		$k' \le k$.
		
		\pause
		\noindent
		\textbf{Running time.}
		The construction can clearly be done in \FPT\ time, and even in polynomial time.
	\end{proof}
	
\end{frame}

\begin{frame}
	\slides{\frametitle{Clique is W[1]-hard II}}

	\begin{corollary}
		\textsc{Clique} is $\W[1]$-hard
	\end{corollary}	

\end{frame}


\section{Further Reading}

\begin{frame}
 \slides{\frametitle{Further Reading}}

  \begin{itemize}
   
   \item Chapter 13, \emph{Fixed-parameter Intractability} in \cite{CyganFKL+15}
%   Marek Cygan, Fedor V.\ Fomin, \L ukasz Kowalik, Daniel Lokshtanov, D{\'a}niel Marx, Marcin Pilipczuk, Micha\l Pilipczuk, and Saket Saurabh. Parameterized Algorithms. Springer, 2015.
   \item Chapter 13, \emph{Parameterized Complexity Theory} in \cite{Niedermeier06}
%   Rolf Niedermeier. Invitation to Fixed Parameter Algorithms. Oxford University Press, 2006.
   \item Elements of Chapters 20--23 in \cite{DowneyF13}
%         Rodney G. Downey and Michael R. Fellows. Fundamentals of Parameterized Complexity. Springer, 2013.
  \end{itemize}

\end{frame}

\begin{frame}[t, allowframebreaks]
\slides{\frametitle{References}}
\printbibliography
\end{frame}


\end{document}
