%!TEX root = 03-kernel-SL.tex

\newcommand{\vcpre}{\textsf{VC-preprocess}\xspace}

\begin{document}

\title[Kernelization]
{Kernelization}

\begin{frame}
	\titlepage
\end{frame}

\lecturenotes{\maketitle}

\begin{frame}
	\slides{\frametitle{Outline}}
	\tableofcontents
\end{frame}


\section{Vertex Cover}

\begin{frame}
	\slides{\frametitle{Vertex cover}}

	A \alert{vertex cover} of a graph $G=(V,E)$ is a subset of vertices $S\subseteq V$ such that
	%each edge of the graph is incident to at least one vertex from $S$, i.e.,
	for each edge $\{u,v\}\in E$, we have $u\in S$ or $v\in S$.

	\pbDef{\VC}{A graph $G=(V,E)$ and an integer $k$}{$k$}{Does $G$ have a \vc of size at most $k$?}

	%Example: $\{b,c,e\}$ is a vertex cover for this graph $H$
	\begin{center}
		\begin{tikzpicture}
			\node[vertex, label=above:$a$]   (a) at (0.5,1.5) {};
			\node[selected, label=above:$b$]   (b) at (0,  1) {};
			\node[selected, label=above:$c$]   (c) at (1,  1) {};
			\node[vertex, label=left:$d$]   (d) at (0,  0) {};
			\node[selected, label=right:$e$]   (e) at (1,  0) {};

			\draw[thick] (d)--(c)--(a)--(b)--(c)--(e)--(b)--(d)--(e);
		\end{tikzpicture}
	\end{center}

\end{frame}


\begin{frame}
	\frametitle{Exercise 1}

	\begin{center}
		\begin{tikzpicture}[scale=2]
			\node[vertex, label=above:$a$]   (a) at (1,2) {};
			\node[vertex, label=above:$b$]   (b) at (0,1) {};
			\node[vertex, label=left:$c$]    (c) at (1,1) {};
			\node[vertex, label=right:$d$]   (d) at (2,1) {};
			\node[vertex, label=above:$e$]   (e) at (0,0) {};
			\node[vertex, label=left:$f$]    (f) at (1,0) {};
			\node[vertex, label=right:$g$]   (g) at (2,0) {};

			\draw[thick] (b)--(a)--(c)--(d)--(a)--(g)--(d)--(f)--(c)--(g)--(f);

			\node at (3,1) {$k=4$};
		\end{tikzpicture}
	\end{center}
	Is this a \Yes-instance for \VC?\\
	(Is there $S\subseteq V$ with $|S|\le 4$, such that $\forall\ uv\in E$, $u\in S$ or $v\in S$?)
\end{frame}

\begin{frame}
	\frametitle{Exercise 2}

	\begin{center}
		\begin{tikzpicture}
			\node[vertex, label=above:$a$]   (a) at (0,7) {};
			\node[vertex, label=above:$b$]   (b) at (2,7) {};
			\node[vertex, label=above:$c$]    (c) at (4,7) {};
			\node[vertex, label=left:$d$]   (d) at (0,6) {};
			\node[vertex, label=above:$e$]   (e) at (2,6) {};
			\node[vertex, label=left:$f$]    (f) at (0,5) {};
			\node[vertex, label=above right:$g$]   (g) at (2,5) {};
			\node[vertex, label=right:$h$]   (h) at (4,5) {};
			\node[vertex, label=right:$i$]   (i) at (2,4) {};
			\node[vertex, label=left:$j$]    (j) at (1,3) {};
			\node[vertex, label=right:$k$]   (k) at (3,3) {};
			\node[vertex, label=left:$l$]   (l) at (2,2) {};
			\node[vertex, label=left:$m$]    (m) at (2,1) {};
			\node[vertex, label=below:$n$]   (n) at (2,0) {};

			\draw[thick] (a)--(d)--(e)--(f)--(d) (h)--(e)--(g)--(h) (f)--(g)--(i)--(f) (n)--(k) (h)--(i)--(k)--(h) (f)--(j)--(i) (j)--(k)--(l)--(j)--(n)--(m)--(l);

			\draw[thick] (e) .. controls +(5,0) and +(3,1) .. (n);
			\draw[thick] (l) .. controls +(0.2,-1) .. (n);
			\draw[thick] (f) .. controls +(0.5,-2.5) .. (n);
			\draw[thick] (h) .. controls +(-0.5,-2.5) .. (n);
			\draw[thick] (f) .. controls +(1.2,-1) .. (k);

			\node at (6,3.5) {$k=7$};
		\end{tikzpicture}
	\end{center}
\end{frame}



\subsection{Simplification rules}

\begin{frame}
	\slides{\frametitle{Simplification rules for \VC}}
	\only<1-3>{
		\begin{block}{(Degree-0)}
			If $\exists v\in V$ such that $d_G(v)=0$, then set $G \leftarrow G-v$.
			%\only<1>{ \footnote[frame]{Notation: $N_G(v) = \{u: \{u,v\}\in E\}$; $N_G[v] = N_G(v)\cup\{v\}$; $d_G(v) = |N_G(v)|$.} \footnote[frame]{For a graph $G=(V,E)$ and a set $S\subseteq V$, the graph $G\setminus S$ is the graph $G[V\setminus S]=(V\setminus S, \{\{u,v\}\in E: u,v\in V\setminus S\})$}}
		\end{block}

		\uncover<2-3>{
			\noindent
			\textbf{Proving correctness.} A simplification rule is \alert{sound} if for every instance, it produces an equivalent instance.
			Two instances $I,I'$ are \alert{equivalent} if they are both \Yes-instances or they are both \No-instances.

			\begin{lemma}
				(Degree-0) is sound.
			\end{lemma}
		}
		\uncover<3>{
			\begin{proof}
				First, suppose $(G-v,k)$ is a \Yes-instance.
				Let $S$ be a vertex cover for $G-v$ of size at most $k$.
				Then, $S$ is also a vertex cover for $G$ since no edge of $G$ is incident to $v$.
				Thus, $(G,k)$ is a \Yes-instance.

				Now, suppose $(G-v,k)$ is a \No-instance.
				For the sake of contradiction, assume $(G,k)$ is a \Yes-instance.
				Let $S$ be a vertex cover for $G$ of size at most $k$.
				But then, $S\setminus \{v\}$ is a vertex cover of size at most $k$ for $G-v$; a contradiction. % contradicting the fact that $(G-v,k)$ is a \No-instance.
			\end{proof}
		}}
	\only<4-5>{
		\begin{block}{(Degree-1)}
			If $\exists v\in V$ such that $d_G(v)=1$, then set $G\leftarrow G- N_G[v]$ and $k\leftarrow k-1$.
		\end{block}

		\uncover<5>{
			\begin{lemma}
				(Degree-1) is sound.
			\end{lemma}
			\begin{proof}
				Let $u$ be the neighbor of $v$ in $G$.
				Thus, $N_G[v] = \{u,v\}$.

				If $S$ is a \vc of $G$ of size at most $k$, then $S\setminus \{u,v\}$ is a \vc of $G- N_G[v]$ of size at most $k-1$, because $u\in S$ or $v\in S$.\\
				If $S'$ is a \vc of $G- N_G[v]$ of size at most $k-1$, then $S'\cup \{u\}$ is a \vc of $G$ of size at most $k$, since all edges that are in $G$ but not in $G- N_G[v]$ are incident to $u$.
			\end{proof}
		}}
	\only<6-7>{
		\begin{block}{(Large Degree)}
			If $\exists v\in V$ such that $d_G(v)>k$, then set $G\leftarrow G- v$ and $k\leftarrow k-1$.
		\end{block}

		\uncover<7>{
			\begin{lemma}
				(Large Degree) is sound.
			\end{lemma}
			\begin{proof}
				Let $S$ be a \vc of $G$ of size at most $k$.
				If $v\notin S$, then $N_G(v)\subseteq S$, contradicting that $|S|\le k$.
			\end{proof}
		}}
	\only<8-9>{
		\begin{block}{(Number of Edges)}
			If $d_G(v)\le k$ for each $v\in V$ and $|E|>k^2$ then return \No
		\end{block}

		\uncover<9>{
			\begin{lemma}
				(Number of Edges) is sound.
			\end{lemma}
			\begin{proof}
				Assume $d_G(v)\le k$ for each $v\in V$ and
				$|E|>k^2$.\\
				Suppose $S\subseteq V$, $|S|\le k$, is a \vc of $G$.\\
				We have that $S$ covers at most $k^2$ edges.\\
				However, $|E|\ge k^2+1$.\\
				Thus, $S$ is not a \vc of $G$.
			\end{proof}
		}}
\end{frame}

\subsection{Preprocessing algorithm}


\begin{frame}
	\slides{\frametitle{Preprocessing algorithm for \VC}}


	\begin{algorithm}[H]
		\DontPrintSemicolon
		{\vcpre}

		\KwIn{A graph $G$ and an integer $k$.}
		\KwOut{A graph $G'$ and an integer $k'$ such that $G$ has a \vc of size at most $k$ if and only if $G'$ has a \vc of size at most $k'$.}
		\SetKwComment{LabRule}{}{}
		\BlankLine
		%\Indp
		$G'\leftarrow G$\;
		$k'\leftarrow k$\;
		\Repeat{no simplification rule applies}{
			Execute simplification rules (Degree-0), (Degree-1), (Large Degree), and (Number of Edges)
			for $(G',k')$
		}
		\Return $(G',k')$
	\end{algorithm}

\end{frame}


\begin{frame}{Effectiveness of preprocessing algorithms}

	\begin{itemize}
		\item How effective is \vcpre?
		\item We would like to study preprocessing algorithms mathematically and quantify their effectiveness.
	\end{itemize}

\end{frame}


\begin{frame}{First try}

	\begin{itemize}
		\item Say that a preprocessing algorithm for a problem $\Pi$ is \alert{nice}
		      if it runs in polynomial time and
		      for each instance for $\Pi$, it returns an instance for $\Pi$ that is strictly smaller.
		      \pause
		\item $\rightarrow$ executing it a linear number of times reduces the instance to a single bit
		\item $\rightarrow$ such an algorithm would solve $\Pi$ in polynomial time
		\item For \NP-hard problems this is not possible unless $\P = \NP$
		\item We need a different measure of effectiveness
	\end{itemize}

\end{frame}


\begin{frame}{Measuring the effectiveness of preprocessing algorithms}
	\begin{itemize}
		\item We will measure the effectiveness in terms of the \alert{parameter}
		\item How large is the resulting instance in terms of the parameter?
	\end{itemize}
\end{frame}


\begin{frame}{Effectiveness of \vcpre}
	\begin{lemma}
		For any instance $(G,k)$ for \VC, \vcpre produces an equivalent instance $(G',k')$ of size $O(k^2)$.
	\end{lemma}
	\begin{proof}
		Since all simplification rules are sound, $(G=(V,E),k)$ and $(G'=(V',E'),k')$ are equivalent.\\
		By (Number of Edges), $|E'|\le (k')^2 \le k^2$.\\
		By (Degree-0) and (Degree-1), each vertex in $V'$ has degree at least 2 in $G'$.\\
		Since $\sum_{v\in V'} d_{G'}(v) = 2 |E'| \le 2k^2$, this implies that $|V'|\le k^2$.\\
		Thus, $|V'|+|E'| \subseteq O(k^2)$.
	\end{proof}
\end{frame}

\section{Kernelization algorithms}

\begin{frame}{Kernelization: definition}

	\begin{definition}
		A \alert{kernelization} for a parameterized problem $\Pi$ is a
		\textbf{polynomial time} algorithm, which, for any instance $I$ of $\Pi$ with parameter $k$,
		produces an \textbf{equivalent} instance $I'$ of $\Pi$ with parameter $k'$ such that
		$|I'| \le f(k)$ and $k'\le f(k)$ for a computable function $f$.\\
		We refer to the function $f$ as the \alert{size} of the kernel.
	\end{definition}

	\noindent
	\textbf{Note}: We do not formally require that $k'\le k$, but this will be the case for many kernelizations.

\end{frame}


\begin{frame}{\vcpre is a quadratic kernelization}
	\begin{theorem}
		\vcpre is a $O(k^2)$ kernelization for \VC.
	\end{theorem}

\end{frame}

\section{Kernel for \HC}

\begin{frame}[allowframebreaks]
	\slides{\frametitle{\HC}}

	\noindent
	A \alert{Hamiltonian cycle} of $G$ is a subgraph of $G$ that is a cycle on $|V(G)|$ vertices.

	\pbDef{\textsf{vc}-\HC}{A graph $G=(V,E)$.}{$k=vc(G)$, the size of a smallest \vc of $G$.}{Does $G$ have a Hamiltonian cycle?}

	\noindent
	\textbf{Thought experiment}: Imagine a very large instance where the parameter is tiny. How can you simplify such an instance?

	\framebreak
	\noindent
	\textbf{Issue}: We do not actually know a \vc of size $k$.\\We do not even know the value of $k$ (it is not part of the input).

	\framebreak

	\begin{itemize}
		\item Obtain a vertex cover using an approximation algorithm. We will use a 2-approximation algorithm, producing a vertex cover of size $\le 2k$ in polynomial time.
		\item If $C$ is a \vc of size $\le 2k$, then $I = V\setminus C$ is an independent set of size $\ge |V|-2k$.
		\item No two consecutive vertices in the Hamiltonian Cycle can be in $I$.
		\item A kernel with $\le 4k$ vertices can now be obtained with the following simplification rule.
	\end{itemize}

	\begin{block}{(Too-large)}
		Compute a vertex cover $C$ of size $\le 2k$ in polynomial time.\\
		If $2|C|<|V|$, then return \No
	\end{block}

\end{frame}


\section{Kernel for \ECC}

\begin{frame}
	\slides{\frametitle{\ECC}}

	\only<1>{
		\begin{definition}
			An \alert{\ecc} of a graph $G=(V,E)$ is a set of cliques in $G$ covering all its edges.\\
			In other words, if $\mathcal{C}\subseteq 2^V$ is an \ecc then each $S\in \mathcal{C}$ is a clique in $G$ and for each $\{u,v\} \in E$ there exists an $S\in \mathcal{C}$ such that $u,v\in S$.
		\end{definition}

		\noindent
		Example: $\{\textcolor{Green}{\{a,b,c\}},\textcolor{Blue}{\{b,c,d,e\}}\}$ is an \ecc for this graph.
		\begin{center}
			\begin{tikzpicture}
				\node[vertex, label=above:$a$]   (a) at (0.5,1.5) {};
				\node[vertex, label=above:$b$]   (b) at (0,  1) {};
				\node[vertex, label=above:$c$]   (c) at (1,  1) {};
				\node[vertex, label=left:$d$]   (d) at (0,  0) {};
				\node[vertex, label=right:$e$]   (e) at (1,  0) {};

				\draw[thick] (d)--(c)--(a)--(b)--(c)--(e)--(b)--(d)--(e);
				\draw[thick,Green,rounded corners=5mm] (0.5,2.3)--(1.7,0.7)--(-0.7,0.7)--cycle;
				\draw[thick,Blue,rounded corners=5mm] (1.6,-0.3)--(-0.6,-0.3)--(-0.6,1.3)--(1.6,1.3)--cycle;
			\end{tikzpicture}
		\end{center}
	}

	\only<2>{
		\pbDef{\ECC}{A graph $G=(V,E)$ and an integer $k$}{$k$}{Does $G$ have an \ecc of size at most $k$?}
		The \alert{size} of an edge clique cover $\mathcal{C}$ is the number of cliques contained in $\mathcal{C}$ and is denoted $|\mathcal{C}|$.
	}
\end{frame}


\begin{frame}{Helpful properties}

	\begin{definition}
		A clique $S$ in a graph $G$ is a \alert{maximal} clique if there is no other clique $S'$ in $G$ with $S\subset S'$.
	\end{definition}

	\begin{lemma}\label{lem:maximalclique}
		A graph $G$ has an \ecc $\mathcal{C}$ of size at most $k$ if and only if $G$ has an \ecc $\mathcal{C}'$ of size at most $k$ such that each $S\in \mathcal{C}'$ is a maximal clique.
	\end{lemma}
	\begin{proof}[Proof sketch]
		$(\Rightarrow)$: Replace each clique $S\in \mathcal{C}$ by a maximal clique $S'$ with $S\subseteq S'$.

		$(\Leftarrow)$: Trivial, since $\mathcal{C}'$ is an \ecc of size at most $k$.
	\end{proof}

\end{frame}



\begin{frame}{Simplification rules for \ECC}

	\noindent
	\textbf{Thought experiment}: Imagine a very large instance where the parameter is tiny. How can you simplify such an instance?

\end{frame}

\begin{frame}
	\slides{\frametitle{Simplification rules for \ECC\ II}}
	\noindent
	The instance could have many degree-0 vertices.

	\begin{block}{(Isolated)}
		If there exists a vertex $v\in V$ with $d_G(v)=0$, then set $G\leftarrow G-v$.
	\end{block}

	\begin{lemma}
		(Isolated) is sound.
	\end{lemma}
	\begin{proof}[Proof sketch]
		Since no edge is incident to $v$, a smallest \ecc\ for $G-v$ is a smallest \ecc\ for $G$, and vice-versa.
	\end{proof}

	\pause

	\begin{block}{(Isolated-Edge)}
		If $\exists uv\in E$ such that $d_G(u)=d_G(v)=1$, then set $G\leftarrow G-\{u,v\}$ and $k\leftarrow k-1$.
	\end{block}
\end{frame}

\begin{frame}
	\slides{\frametitle{Simplification rules for \ECC\ III}}

	\begin{block}{(Twins)}
		If $\exists u,v\in V$, $u\ne v$, such that $N_G[u]=N_G[v]$, then set $G\leftarrow G-v$.
	\end{block}

	\begin{lemma}
		(Twins) is sound.
	\end{lemma}
	\pause
	\begin{proof}
		We need to show that $G$ has an \ecc of size at most $k$ if and only if $G-v$ has an \ecc of size at most $k$.

		$(\Rightarrow)$: If $\mathcal{C}$ is an \ecc of $G$ of size at most $k$, then $\{S\setminus \{v\} : S\in \mathcal{C}\}$ is an \ecc of $G-v$ of size at most $k$.

		$(\Leftarrow)$: Let $\mathcal{C}'$ be an \ecc of $G-v$ of size at most $k$. Partition $\mathcal{C}'$ into $\mathcal{C}'_u = \{S\in \mathcal{C}': u\in S\}$ and $\mathcal{C}'_{\neg u} = \mathcal{C}' \setminus \mathcal{C}'_u$. Note that each set in $\mathcal{C}_{u}=\{S\cup \{v\} : S\in \mathcal{C}'_u\}$ is a clique in $G$ since $N_G[u]=N_G[v]$ and that each edge incident to $v$ is contained in at least one of these cliques. Now, $\mathcal{C}_{u} \cup \mathcal{C}'_{\neg u}$ is an \ecc of $G$ of size at most $k$.
	\end{proof}
\end{frame}

\begin{frame}
	\slides{\frametitle{Simplification rules for \ECC\ IV}}

	\begin{block}{(Size-V)}
		If the previous simplification rules do not apply and $|V|>2^k$, then return \No.
	\end{block}

	\begin{lemma}
		(Size-V) is sound.
	\end{lemma}
	\pause
	\begin{proof}
		For the sake of contradiction, assume neither (Isolated) nor (Twins) are applicable, $|V|>2^k$, and $G$ has
		an \ecc $\mathcal{C}$ of size at most $k$. Since $2^{\mathcal{C}}$ (the set of all subsets of $\mathcal{C}$) has
		size at most $2^k$, and every vertex belongs to at least one clique in $\mathcal{C}$ by (Isolated), we have that there exists two vertices $u,v\in V$ such that $\{S\in \mathcal{C} : u\in S\} = \{S\in \mathcal{C} : v\in S\}$. But then,
		$N_G[u]= \bigcup_{S\in \mathcal{C} : u\in S} S = \bigcup_{S\in \mathcal{C} : v\in S} S = N_G[v]$, contradicting that (Twin) is not applicable.
	\end{proof}

\end{frame}

\begin{frame}
	\frametitle{Kernel for \ECC}

	\begin{theorem}[\cite{GrammGHN08}]
		\ECC\ has a kernel with $O(2^k)$ vertices and $O(4^k)$ edges.
	\end{theorem}

	\begin{corollary}
		\ECC\ is \FPT.
	\end{corollary}

\end{frame}

\section{Kernels and Fixed-parameter tractability}

\begin{frame}
	\slides{\frametitle{Kernels and Fixed-parameter tractability}}

	\begin{theorem}
		Let $\Pi$ be a decidable parameterized problem.\\
		$\Pi$ has a kernelization algorithm $\Leftrightarrow$ $\Pi$ is \FPT.
	\end{theorem}
	\pause
	\begin{proof}
		$(\Rightarrow)$: An \FPT\ algorithm is obtained by first running the kernelization, and then any brute-force algorithm on the resulting instance.

		$(\Leftarrow)$: Let $A$ be an \FPT\ algorithm for $\Pi$ with running time $O(f(k) n^c)$.\\
		If $f(k)<n$, then $A$ has running time $O(n^{c+1})$. In this case, the kernelization algorithm runs $A$ and returns a trivial \Yes- or \No-instance depending on the answer of $A$.\\
		Otherwise, $f(k)\ge n$. In this case, the kernelization algorithm outputs the input instance.
	\end{proof}

\end{frame}



\section{Further Reading}

\begin{frame}
	\slides{\frametitle{Further Reading}}

	\begin{itemize}
		\item Chapter 2, \emph{Kernelization} in \cite{CyganFKL+15}
		\item Chapter 4, \emph{Kernelization} in \cite{DowneyF13}
		\item Chapter 7, \emph{Data Reduction and Problem Kernels} in \cite{Niedermeier06}
		\item Chapter 9, \emph{Kernelization and Linear Programming Techniques} in \cite{FlumG06}
		\item the kernelization book \cite{FominLSZ19}
	\end{itemize}

\end{frame}


\begin{frame}[t]
	\slides{\frametitle{References}}
	\printbibliography
\end{frame}

\end{document}
