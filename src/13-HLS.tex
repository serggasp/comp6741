%!TEX root = ../out/13-HLS-SL.tex

\begin{document}

\mytitle{Heuristics and Local Search}{Heuristics and Local Search}

\section{Introduction}

\begin{frame}
	\frametitle{Optimisation problems}

  Recall:
  	\begin{definition}
  		An \alert{optimisation problem} is characterised by
  		\begin{itemize}
  			\item a set of input instances
  			\item a set of \alert{feasible solutions} for each input instance
  			\item a \alert{value} for each feasible solution
  		\end{itemize}
  		In a \alert{maximisation} problem (resp., a \alert{minimisation}) problem, the goal is to find a feasible solution with maximum (resp., minimum) value.
  	\end{definition}
  	
  	\noindent \textbf{Example:}
  	In the \textsc{Graph Coloring} minimisation problem, the input is a graph $G$, the feasible solutions are all the proper colorings of $G$, and the value of a coloring is its number of colors.

\end{frame}

\begin{frame}
	\frametitle{Heuristics}
	
	\begin{itemize}
		\item return a feasible solution
		\item typically, there is no guarantee on the value of the solution returned by the heuristic
		\item ideally, they are fast, easy to implement, and return satisfactory solutions for many instances
	\end{itemize}
\end{frame}

\begin{frame}
	\frametitle{Local Search}
	
	\begin{itemize}
		\item Given a feasible solution $S$, a local search algorithm explores whether there is an improved feasible solution $S^*$ that can be obtained from $S$ by a ``small modification''.
	\end{itemize}
\end{frame}


\section{Case study: graph coloring}

\begin{frame}
	Discussion on \cite{HartungN13}.
	
	\slides{\vspace{8cm}}
\end{frame}

\section{Case study: treewidth}

\begin{frame}
	Discussion on \cite{GaspersGJMR19}.
	
	\slides{\vspace{8cm}}
\end{frame}


\begin{frame}[t, allowframebreaks]
	\slides{\frametitle{References}}
	\printbibliography
\end{frame}


\end{document}
